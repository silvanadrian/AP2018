\documentclass[12pt,a4paper]{article}
\usepackage[english, science, large]{../ku-frontpage}
\usepackage{tabularx}
\usepackage{ltablex}
\usepackage{minted}
\setminted[haskell]{
frame=lines,
framesep=2mm,
baselinestretch=1.1,
fontsize=\footnotesize,
linenos,
breaklines}
\hypersetup{
    colorlinks=false,
    pdfborder={0 0 0},
}
\begin{document}

\title{Haskell intro}
\subtitle{Assignement0}

\author{Kai Arne S. Myklebust, Silvan Adrian}
\date{Handed in: \today}
	
\maketitle
\tableofcontents

\section{Design/Implementation}

Overall in this assignment our goal was to not use helper functions, where not especially needed. This to make readability easier and making the code less complex since many of our helper functions where unnecesarry.
But, we started of by making a helper function for each arithmetic operation from "initial context". This was to get the first and second element from the list. After a while we got really annoyd doing that and found out that you can just get the first and second element by changing to [] list-brackets.

In evalExpr we used the do notation to

\section{Code Assessment}

equality mit Arrays Edge Case war nicht abgedeckt ([] a, a [])


\appendix
\section{Code Listing}

\inputminted{haskell}{assignement1/src/SubsInterpreter.hs}


\end{document}}