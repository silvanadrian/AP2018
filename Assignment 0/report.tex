\documentclass[12pt,a4paper]{article}
\usepackage[english, science, large]{../ku-frontpage}
\usepackage{tabularx}
\usepackage{ltablex}
\usepackage{minted}
\setminted[haskell]{
frame=lines,
framesep=2mm,
baselinestretch=1.1,
fontsize=\footnotesize,
linenos,
breaklines}
\hypersetup{
    colorlinks=false,
    pdfborder={0 0 0},
}
\begin{document}

\title{Haskell intro}
\subtitle{Assignement0}

\author{Kai Arne S. Myklebust, Silvan Adrian}
\date{Handed in: \today}
	
\maketitle
\tableofcontents

\section{Design/Implementation}
We always tried to move as much code as possible to own functions that the code doesn't get too unreadable.
For Example the function `showExpr` would have had too much duplicated code which we then refactored out.
Also the function `summ` is a good example for making it more readable by moving the functionality out of `evalFull` and only call the function from there.
This also helps with reusability overall in the code in case one of the functions can be used many times (like `summ` in `evalFull` and `evalError`).

Additionally we also didn't check for division by zero in `evalSimple` or `evalFull` since haskell takes care of those errors.

We also tried to use as much as possible out of the standard library, for example from \textbf{Data.Either} the `isRight` function. Otherwise we would have had to implement it ourselves (or end up doing something totally different). For `fromRight` we decided to implement it ourselves as `fromRight'`, in which we don't have to pass any default value and return an error in case it's not a `Right Either`.

Overall we tried to keep it as simple as possible and declutter code wherever it was possible. \newline

\textbf{Edit 17. September:}
We had to use eager evaluation for the single case of Negative Power Exponent since otherwise Haskell tries to be intelligent and just ignores part of the expression.
That's why we use `seq`to be sure that `Pow` gets evaluated rightly.
The rest is lazy evaluated.



\section{Code Assessment}

By moving some functionality into own functions we do believe we increased the maintainability at least in some parts, especially when you don't have to change the code in many places.

We also try to handle all kind of edge cases as good as possible that the code should be able to handle errors or wrong inputs (either by error message or haskell error depending on which eval function). As said, for example with division by zero haskell catches that error itself, so we did not find it necessary to write our own test for that.

Additionally we sometimes ended up with long lines (longer then 80 Chars) which might not seem that nice, but for the sake of having a one line solution it was a necessary evil. Sum is the worst example in this regard, but it needs so many parameters that need to be checked so it was difficult do to it shorter.


We also wrote some tests which can be run by `stack test`, which test the basic functionality to our best knowledge.

\appendix
\section{Code Listing}

\inputminted{haskell}{src/Arithmetic.hs}


\end{document}}