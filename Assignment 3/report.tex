\documentclass[12pt,a4paper]{article}
\usepackage[english, science, large]{../template/ku-frontpage}
\usepackage{tabularx}
\usepackage{ltablex}
\usepackage{minted}
\setminted[haskell]{
frame=lines,
framesep=2mm,
baselinestretch=1.1,
fontsize=\footnotesize,
linenos,
breaklines}
\usemintedstyle{friendly}
\hypersetup{
    colorlinks=false,
    pdfborder={0 0 0},
}
\begin{document}

\title{Haskell intro}
\subtitle{Assignment 2}

\author{Kai Arne S. Myklebust, Silvan Adrian}
\date{Handed in: \today}
	
\maketitle
\tableofcontents

\section{Design/Implementation}

We implemented all needed predicates by not using any built-in Prolog predicates or control operators, which we were restricted to use.
So we had to implement predicates for select and not ourselves. In addition we defined other helper predicates to solve the tasks.

\section{Code Assessment}
We wrote Tests for testing all the defined predicates, which test either negative or positive responses of them.
Also we used 2 different Lists to be able to test the predicates on 2 Lists.

OnlineTA also returned no errors by testing out our predicates.

We needed to implement a lot of new helper predicates. We tried to give them good names and place them after they were used, but sometimes it could get a little bit confusing what does what and where everything is defined.

\appendix
\section{Code Listing}

\inputminted{prolog}{handin/twitbook.pl}

\section{Tests Listing}

\inputminted{prolog}{handin/tests.pl}



\end{document}}