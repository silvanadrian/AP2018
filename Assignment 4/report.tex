\documentclass[12pt,a4paper]{article}
\usepackage[english, science, large]{../template/ku-frontpage}
\usepackage{tabularx}
\usepackage{ltablex}
\usepackage[cache=false]{minted}
\setminted[haskell]{
frame=lines,
framesep=2mm,
baselinestretch=1.1,
fontsize=\footnotesize,
linenos,
breaklines}
\usemintedstyle{friendly}
\hypersetup{
    colorlinks=false,
    pdfborder={0 0 0},
}
\begin{document}

\title{The Flamingo Route}
\subtitle{Assignment 4}

\author{Kai Arne S. Myklebust, Silvan Adrian}
\date{Handed in: \today}
	
\maketitle
\tableofcontents

\section{Solution}

\subsection{Files}
All Files are situated in the \textbf{src/} folder:
\begin{itemize}
	\item \textbf{flamingo.erl} The flamingo server implmentation
	\item \textbf{greetings.erl} The greetings module implementation
	\item \textbf{hello.erl} The hello module implementation
	\item \textbf{test\_flamingo.erl} the file which includes all the tests
\end{itemize}

\subsection{Running the programm}
Out of convenience we used a Emakefile which compiles all the erlang files in one go then rather compile each file on it's own.
This can be done by using the erlang shell and run:

\begin{minted}{erlang}
	make:all([load]).
\end{minted}

\subsection{Running the tests}
The tests can be run with eunit, we included all test for all modules in the same file for convenience.

\begin{minted}{erlang}
	eunit:test(test\_flamingo, [verbose]).
\end{minted}

\section{Implementation}

\subsection{Assumptions}

\subsection{Possible improvements}

\section{Assessment}

\subsection{Scope of Test Cases}

\subsection{Correctness}

\subsection{Code Quality}

\appendix
\section{Code Listing}

\section{Tests Listing}

\end{document}}